% $Id$
%


\usebackgroundtemplate{\includegraphics[width=13cm]{figs/candado}}
{\bf
  \textcolor[rgb]{1,1,1}{
    \section{Presentaci�n}
  }
}

\usebackgroundtemplate{}

%%---------------------------------------------------------------

\begin{frame}
\frametitle{Pr�cticas educativas abiertas}

\includegraphics[width=6.5cm]{figs/abierto-todo}

{\Large
\begin{itemize}
\item Abiertas hacia los alumnos:
  \begin{flushright}
    nuevos roles, m�s activos \\
    aprovechar las habilidades del alumno \\
  \end{flushright}

    
\item Abiertas hacia los dem�s:
  \begin{flushright}
    trascender las paredes del aula
    aprovechar Internet
  \end{flushright}
\end{itemize}
}
\end{frame}

%%---------------------------------------------------------------

\begin{frame}
\frametitle{Trasladando desde el software libre}

{\Large
\begin{itemize}
\item Abiertas hacia los alumnos:
  \begin{flushright}
    aprendizaje con ayuda de pares \\
    difuminaci�n de papeles \\ (profesor / alumno) \\
  \end{flushright}

\vspace{1cm}
\item Abiertas hacia los dem�s:
  \begin{flushright}
    escrutinio p�blico \\
    creaci�n de comunidad \\
  \end{flushright}
\end{itemize}
}
\end{frame}

%%---------------------------------------------------------------

\begin{frame}
\frametitle{Hacia los alumnos}

{\Large
Actores principales del aprendizaje

\begin{itemize}
\item Tienen muchas habilidades muy �tiles
\item Saben muy bien qu� les falta \\
  (incluso cuando no lo saben) \\
\item Comprenden muy bien a otros alumnos
\item El profesor atento a
  completar, corregir, comentar \\
\end{itemize}

\begin{flushright}
  M�s posibilidades conllevan m�s responsabilidades
\end{flushright}
}

\end{frame}

%%---------------------------------------------------------------

\begin{frame}
\frametitle{Hacia los dem�s}

{\Large

Hay mucha gente aprendiendo

\begin{itemize}
\item Sin coste a�adido se llega a m�s gente
\item Potencialmente, m�s gente puede colaborar
\item Podemos reutilizar de muchos \\
  y muchos pueden reutilizar lo nuestro \\
\item El potencial de convertirse en referencia
\item El potencial de conseguir colaboradores
\end{itemize}

\begin{flushright}
  Los beneficios de escala pueden ser grandes
\end{flushright}
}

\end{frame}

%%---------------------------------------------------------------

\begin{frame}
\frametitle{Estructura del m�dulo}

{\Large
Metodolog�a: exposici�n de casos

\begin{itemize}
\item Apuntes en colaboraci�n
\item Glosario de asignatura
\item Documentos libres
\item Escritura en Wikipedia  
\end{itemize}

\vspace{1cm}

Debate: Asignaturas en abierto: �riesgos, ventajas?
}

\end{frame}

%%---------------------------------------------------------------

\begin{frame}
\frametitle{Este m�dulo ser� (un poco) abierto}

{\Large

\begin{itemize}
\item Materiales disponibles
\item Mejoras / comentarios bienvenidos
\item �Alguien quiere colaborar con apuntes abiertos?
\item Debate, discusi�n en el el foro
\end{itemize}

\vspace{1cm}

\begin{center}
�Si quieres colaborar, dinos!
\end{center}
}

\end{frame}

