% $Id$
%


\usebackgroundtemplate{\includegraphics[width=13cm]{figs/acta}}
{\bf
  \textcolor[rgb]{1,1,1}{
    \section{Apuntes en colaboraci�n}
  }
}

\usebackgroundtemplate{}

%%---------------------------------------------------------------

\begin{frame}
\frametitle{Y al principio fue el c�digo...}

Pr�cticas de la comunidad del software libre:

{\Large
\begin{itemize}
\item ``Mu�strame el c�digo fuente''
\item Aprendizaje leyendo el c�digo de otros
\item Aprendizaje modificando, incluso estropeando el c�digo de otros
\item Co-aprendizaje trabajando en el mismo c�digo
\item Aprendizaje escribiendo c�digo, y recibiendo cr�ticas
\item Mejora incremental por revisi�n de pares
\end{itemize}
}
\end{frame}

%%---------------------------------------------------------------

\begin{frame}
\frametitle{El patr�n}

{\Large
Primero leer, luego modificar, luego escribir
\begin{flushright}
...y aprender mientras tanto.
\end{flushright}

\vspace{.5cm}

Escuchado a muchos educadores:

\begin{itemize}
\item cuando lees, aprendes algo
\item cuando escribes, aprendes m�s
\item cuando explicas, es cuando aprendes de verdad
\end{itemize}
}
\end{frame}

%%---------------------------------------------------------------

\begin{frame}
\frametitle{La idea}

{\Large
Que los alumnos mejoren su aprendizaje escribiendo apuntes en colaboraci�n

\begin{itemize}
\item Los apuntes de los alumnos pueden ser los mejores apuntes
\item Alta motivaci�n para escribir buenos apuntes para ellos mismos
\item Los alumnos de la misma asignatura escriben sobre los mismos temas
\item Los port�tiles, las grabadoras facilitan un proceso completamente digital 
\end{itemize}

}

\end{frame}

%%---------------------------------------------------------------

\begin{frame}
\frametitle{Posibilidades t�cnicas}

{\Large

\begin{itemize}
\item Procesador de texto, subiendo resultados a Moodle
\item Wikis (integrados en Moodle o no).
\item Editores colaborativos, como Google Docs.
\item Sistemas de control de versiones, como git
\item Editores colaborativos basados en control de versiones, como GitBook
\end{itemize}

\begin{flushright}
  \url{https://www.google.com/docs/about/} \\
  \url{https://github.com/} \\
  \url{https://www.gitbook.com/} \\
\end{flushright}
}

\end{frame}

