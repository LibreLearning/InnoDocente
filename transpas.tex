% main.tex
% Fichero principal de transparencias (incluye a todos los dem�s).

% Compilar a .pdf con LaTeX (pdflatex)
% Es necesario instalar Beamer (paquete latex-beamer en Debian)
%

% Gr�ficos:
% Los gr�ficos pueden suministrarse en PNG, JPG, TIF, PDF, MPS
% Los EPS deben convertirse a PDF (usar epstopdf)
%
\documentclass{beamer}
\usetheme{Warsaw}
\beamertemplatenavigationsymbolsempty
\setbeamertemplate{headline}{}
\useoutertheme{infolines}
%\usebackgroundtemplate{\includegraphics[width=\paperwidth]{format/libresoft-bg-soft.png}}
\usepackage[spanish]{babel}
\usepackage[latin1]{inputenc}
\usepackage{graphics}
\usepackage{amssymb} % Simbolos matematicos

\newcommand{\asignatura}{Pr�cticas educativas abiertas}
\newcommand{\grado}{Itinerario formativo en innovaci�n did�ctica (URJC)}

%\definecolor{libresoftgreen}{RGB}{162,190,43}
%\definecolor{libresoftblue}{RGB}{0,98,143}

%\setbeamercolor{titlelike}{bg=libresoftgreen}

%% Metadatos del PDF.
\hypersetup{
  pdftitle={\asignatura},
  pdfauthor={Jes�s M. Gonz�lez Barahona, Gregorio Robles},
  pdfcreator={GSyC, Universidad Rey Juan Carlos},
  pdfproducer=PDFLaTeX,
  pdfsubject={\asignatura},
}
%%


%%%%%%%%%%%%%%%%%%%%%%%%%%%%%%%%%%%%%%%%%%%%%%%%%%%%%%%%%%%%%%%%
%%%%%%%%%%%%%%%%%%%%%%%%%%%%%%%%%%%%%%%%%%%%%%%%%%%%%%%%%%%%%%%%
% include-only                                                 %
%%%%%%%%%%%%%%%%%%%%%%%%%%%%%%%%%%%%%%%%%%%%%%%%%%%%%%%%%%%%%%%%
%%%%%%%%%%%%%%%%%%%%%%%%%%%%%%%%%%%%%%%%%%%%%%%%%%%%%%%%%%%%%%%%
%\includeonly{intro}

\AtBeginSection[]
{
\begin{frame}<beamer>
\begin{center}
{\Huge \insertsection}
\end{center}
\end{frame}
}


\begin{document}

\title[\asignatura]{\asignatura}
\subtitle{\grado}
\author[GSyC]{Jes�s M. Gonz�lez Barahona, Gregorio Robles Mart�nez}
\institute{GSyC, Universidad Rey Juan Carlos}

\date{Junio 2015}


\frame{
\maketitle

\begin{center}
\includegraphics[width=6cm]{format/gsyc-urjc}
\end{center}
}


% Si el titulo o el autor se quieren acortar para los pies de p�gina
% se pueden redefinir aqu�:
%\title{Titulo corto}
%\author{Autores abreviado}


%% LICENCIA DE REDISTRIBUCION DE LAS TRANSPAS
\frame{
~
\vspace{3cm}

\begin{flushright}

\includegraphics[width=2.2cm]{format/by-sa}
 \\

\begin{footnotesize}
\copyright 2014-2015 Jes�s M. Gonz�lez Barahona y Gregorio Robles. \\

Algunos derechos reservados. Este art�culo se distribuye bajo
la licencia ``Reconocimiento-CompartirIgual 3.0 Espa�a'' de Creative Commons,
disponible en \\
{\small \url{http://creativecommons.org/licenses/by-sa/3.0/es/deed.es}}

Este documento (o uno muy similar) est� disponible en \\
\url{http://LibreLearning.github.io}

\end{footnotesize}
\end{flushright}

}
%%

\frame{
\tableofcontents
}

%%%%%%%%%%%%%%%%%%%%%%%%%%%%%%%%%%%%%%%%%%%%%%%%%%%%%%%%%%%%%%%%
%%%%%%%%%%%%%%%%%%%%%%%%%%%%%%%%%%%%%%%%%%%%%%%%%%%%%%%%%%%%%%%%
% lista de temas                                               %
%%%%%%%%%%%%%%%%%%%%%%%%%%%%%%%%%%%%%%%%%%%%%%%%%%%%%%%%%%%%%%%%
%%%%%%%%%%%%%%%%%%%%%%%%%%%%%%%%%%%%%%%%%%%%%%%%%%%%%%%%%%%%%%%%
% $Id$
%


\usebackgroundtemplate{\includegraphics[width=13cm]{figs/candado}}
{\bf
  \textcolor[rgb]{1,1,1}{
    \section{Presentaci�n}
  }
}

\usebackgroundtemplate{}

%%---------------------------------------------------------------

\begin{frame}
\frametitle{Pr�cticas educativas abiertas}

\includegraphics[width=6.5cm]{figs/abierto-todo}

{\Large
\begin{itemize}
\item Abiertas hacia los alumnos:
  \begin{flushright}
    nuevos roles, m�s activos \\
    aprovechar las habilidades del alumno \\
  \end{flushright}

    
\item Abiertas hacia los dem�s:
  \begin{flushright}
    trascender las paredes del aula
    aprovechar Internet
  \end{flushright}
\end{itemize}
}
\end{frame}

%%---------------------------------------------------------------

\begin{frame}
\frametitle{Trasladando desde el software libre}

{\Large
\begin{itemize}
\item Abiertas hacia los alumnos:
  \begin{flushright}
    aprendizaje con ayuda de pares \\
    difuminaci�n de papeles \\ (profesor / alumno) \\
  \end{flushright}

\vspace{1cm}
\item Abiertas hacia los dem�s:
  \begin{flushright}
    escrutinio p�blico \\
    creaci�n de comunidad \\
  \end{flushright}
\end{itemize}
}
\end{frame}

%%---------------------------------------------------------------

\begin{frame}
\frametitle{Hacia los alumnos}

{\Large
Actores principales del aprendizaje

\begin{itemize}
\item Tienen muchas habilidades muy �tiles
\item Saben muy bien qu� les falta \\
  (incluso cuando no lo saben) \\
\item Comprenden muy bien a otros alumnos
\item El profesor atento a
  completar, corregir, comentar \\
\end{itemize}

\begin{flushright}
  M�s posibilidades conllevan m�s responsabilidades
\end{flushright}
}

\end{frame}

%%---------------------------------------------------------------

\begin{frame}
\frametitle{Hacia los dem�s}

{\Large

Hay mucha gente aprendiendo

\begin{itemize}
\item Sin coste a�adido se llega a m�s gente
\item Potencialmente, m�s gente puede colaborar
\item Podemos reutilizar de muchos \\
  y muchos pueden reutilizar lo nuestro \\
\item El potencial de convertirse en referencia
\item El potencial de conseguir colaboradores
\end{itemize}

\begin{flushright}
  Los beneficios de escala pueden ser grandes
\end{flushright}
}

\end{frame}

%%---------------------------------------------------------------

\begin{frame}
\frametitle{Estructura del m�dulo}

{\Large
Metodolog�a: exposici�n de casos

\begin{itemize}
\item Apuntes en colaboraci�n
\item Glosario de asignatura
\item Documentos libres
\item Escritura en Wikipedia  
\end{itemize}

\vspace{1cm}

Debate: Asignaturas en abierto: �riesgos, ventajas?
}

\end{frame}

%%---------------------------------------------------------------

\begin{frame}
\frametitle{Este m�dulo ser� (un poco) abierto}

{\Large

\begin{itemize}
\item Materiales disponibles
\item Mejoras / comentarios bienvenidos
\item �Alguien quiere colaborar con apuntes abiertos?
\item Debate, discusi�n en el el foro
\end{itemize}

\vspace{1cm}

\begin{center}
�Si quieres colaborar, dinos!
\end{center}
}

\end{frame}


% $Id$
%


\usebackgroundtemplate{\includegraphics[width=13cm]{figs/acta}}
{\bf
  \textcolor[rgb]{1,1,1}{
    \section{Apuntes en colaboraci�n}
  }
}

\usebackgroundtemplate{}

%%---------------------------------------------------------------

\begin{frame}
\frametitle{Y al principio fue el c�digo...}

Pr�cticas de la comunidad del software libre:

{\Large
\begin{itemize}
\item ``Mu�strame el c�digo fuente''
\item Aprendizaje leyendo el c�digo de otros
\item Aprendizaje modificando, incluso estropeando el c�digo de otros
\item Co-aprendizaje trabajando en el mismo c�digo
\item Aprendizaje escribiendo c�digo, y recibiendo cr�ticas
\item Mejora incremental por revisi�n de pares
\end{itemize}
}
\end{frame}

%%---------------------------------------------------------------

\begin{frame}
\frametitle{El patr�n}

{\Large
Primero leer, luego modificar, luego escribir
\begin{flushright}
...y aprender mientras tanto.
\end{flushright}

\vspace{.5cm}

Escuchado a muchos educadores:

\begin{itemize}
\item cuando lees, aprendes algo
\item cuando escribes, aprendes m�s
\item cuando explicas, es cuando aprendes de verdad
\end{itemize}
}
\end{frame}

%%---------------------------------------------------------------

\begin{frame}
\frametitle{La idea}

{\Large
Que los alumnos mejoren su aprendizaje escribiendo apuntes en colaboraci�n

\begin{itemize}
\item Los apuntes de los alumnos pueden ser los mejores apuntes
\item Alta motivaci�n para escribir buenos apuntes para ellos mismos
\item Los alumnos de la misma asignatura escriben sobre los mismos temas
\item Los port�tiles, las grabadoras facilitan un proceso completamente digital 
\end{itemize}

}

\end{frame}

%%---------------------------------------------------------------

\begin{frame}
\frametitle{Posibilidades t�cnicas}

{\Large

\begin{itemize}
\item Procesador de texto, subiendo resultados a Moodle
\item Wikis (integrados en Moodle o no).
\item Editores colaborativos, como Google Docs.
\item Sistemas de control de versiones, como git
\item Editores colaborativos basados en control de versiones, como GitBook
\end{itemize}

\begin{flushright}
  \url{https://www.google.com/docs/about/} \\
  \url{https://github.com/} \\
  \url{https://www.gitbook.com/} \\
\end{flushright}
}

\end{frame}


% $Id$
%


%\usebackgroundtemplate{\includegraphics[width=13cm]{figs/biblia}}
%{\bf
%  \textcolor[rgb]{1,1,1}{
    \section{Glosario}
%  }
%}

%\usebackgroundtemplate{}

%%---------------------------------------------------------------

\begin{frame}
\frametitle{Glosarios en comunidad}

{\Large
\begin{itemize}
\item Moodle permite construir glosarios
\item Los alumnos pueden enviar definiciones de t�rminos
\item Las definiciones pueden ser muy completas, incluyendo enlaces, adjuntos...
\item Los alumnos pueden poner comentarios
\item Profesores y alumnos pueden evaluar definiciones
\item Muy interesante para familiazarse con un vocabulario
\end{itemize}
}
\end{frame}





% $Id$
%


\usebackgroundtemplate{\includegraphics[width=13cm]{figs/biblia}}
{\bf
  \textcolor[rgb]{1,1,1}{
    \section{Documentos libres}
  }
}

\usebackgroundtemplate{}

%%---------------------------------------------------------------

\begin{frame}
\frametitle{Una excusa para promover una comunidad}

{\Large
\begin{itemize}
\item Un documento puede promover su propia ``comunidad de inter�s''
\item Si el documento es libre, la comunidad puede contribuir a su mejora
\item La comunidad puede no actuar de forma coordinada, ni en el mismo momento
\item El documento puede evolucionar de muchas formas (traducciones, actualizaciones, mezclas, mejoras)
\item En general, el documento conseguir� mucha m�s visbilidad.
\end{itemize}
}
\end{frame}

%%---------------------------------------------------------------

\begin{frame}
\frametitle{Requisitos b�sicos}

{\Large
Utilizar, modificar, redistribuir
\begin{flushright}
  ...aprovechar a otros, aprovecharse de otros
\end{flushright}

\vspace{.5cm}

\begin{itemize}
\item permisos legales (por el/los autores)
\item posibilidades t�cnicas
\end{itemize}
}
\end{frame}

%%---------------------------------------------------------------

\begin{frame}
\frametitle{Obras culturales libres}

{\Large

  Garant�a de las libertades b�sicas:
\begin{itemize}
\item Libertad de usar la obra
\item Libertad de estudiar la obra y aplicar lo aprendido
\item Libertad de hacer y redistribuir copias
\item Libertad de hacer cambios y mejoras
\end{itemize}

\includegraphics[width=3cm]{figs/free-cultural-works}

\begin{flushright}
  \url{http://freedomdefined.org/Definition/Es}
\end{flushright}
}

\end{frame}

%%---------------------------------------------------------------

\begin{frame}
\frametitle{Licencias libres}

{\Large

\begin{itemize}
\item Creative Commons Atribuci�n
\item Creative Commons Atribuci�n Compartir Igual
\item Obras en el dominio p�blico
\end{itemize}

\begin{flushright}
  \url{http://freedomdefined.org/Licenses} \\
  \url{http://creativecommons.org/freeworks} \\
\end{flushright}
}

\end{frame}

%%---------------------------------------------------------------

\begin{frame}
\frametitle{Consejos pr�cticos}

{\Large

\begin{itemize}
\item Utiliza una licencia libre y \\
  marca claramente la obra con ella
\item Reconoce las licencias (y la autor�a) \\
  de las obras que reutilices
\item Deja versiones modificables de la obra \\
  (editables si es texto, por ejemplo)
\end{itemize}

\begin{flushright}
  \url{http://freedomdefined.org/Licenses}
\end{flushright}
}

\end{frame}

%%---------------------------------------------------------------

\begin{frame}
\frametitle{Conseguir recursos libres (ejemplos:im�genes)}

{\Large

\begin{itemize}
\item Colecciones libres de Flickr \\
  (b�squeda avanzada, por licencia)
\item Wikimedia Commons
\item Google Images \\
  (herramientas de b�squeda, derechos de uso)
\end{itemize}

\begin{flushright}
  \url{http://flickr.com} \\
  \url{http://commons.wikimedia.org} \\
  \url{http://google.com} \\
\end{flushright}
}

\end{frame}




% $Id$
%


%\usebackgroundtemplate{\includegraphics[width=13cm]{figs/biblia}}
%{\bf
%  \textcolor[rgb]{1,1,1}{
    \section{Edici�n en Wikipedia}
%  }
%}

%\usebackgroundtemplate{}

%%---------------------------------------------------------------

\begin{frame}
\frametitle{Colaboraci�n con una comunidad}

{\Large
\begin{itemize}
\item Creaci�n o mejora de art�culos en Wikipedia
\item Muy importante: conocer y respetar las normas de la comunidad
\item Gran impacto fuera del grupo de aprendizaje
\item Incentivo: colaboraci�n en un proyecto muy conocido
\item Atenci�n: Wikipedia es una enciclopedia \\
  �hay que tenerlo en cuenta!
\end{itemize}
}
\end{frame}






%%---------------------------------------------------------------

\begin{frame}
\frametitle{Cr�ditos}

{\footnotesize
\begin{itemize}
\item Rompiendo ataduras, por Fran Silva (Flickr) \\
  Creative Commons Atribuci�n 2.0 \\
  \url{https://www.flickr.com/photos/iscopy/6172392460}
  % candado.jpg
\item Abierto todo el a�o, por Montecruz Foto (Flickr) \\
  Creative Commons Atribuci�n Compartir Igual 2.0 \\
  \url{https://www.flickr.com/photos/libertinus/982246782/}
  % abierto-todo.jpg
\item Acta de la independencia de Bolivia, por Anakin (Wikimedia Commons) \\
  Creative Commons Attribution Share Alike 3.0 Unported \\
  \url{http://commons.wikimedia.org/wiki/File:Indepedence_treaty_of_Bolivia.jpg}
  % acta.jpg
\item Biblia latina mostrada en Malmesbury Abbey, Wiltshire, Inglaterra. Escrita a mano por Gerard Brils, fotografiada por Adrian Pingstone (Wikimedia Commons)\\
  Dominio p�blico \\
  \url{http://commons.wikimedia.org/wiki/File:Illuminated.bible.closeup.arp.jpg}
  % biblia.jpg
\item Logo de ``Free Cultural Works'', por Marc Falzon (Wikimedia Commons) \\
  Dominio p�blico \\
  \url{http://commons.wikimedia.org/wiki/File:Definition_of_Free_Cultural_Works_logo_notext.svg}
  % free-cultural-works.png
\end{itemize}
}

\end{frame}

\end{document}
